\documentclass[a4paper, notitlepage]{article}
\usepackage{fullpage, listings, courier}
\usepackage{float}
\usepackage[pdftex]{graphicx}
\lstset{language=Java,basicstyle=\small\ttfamily,commentstyle=\color{Gray},tabsize=4}

\begin{document}

\title{IN4325 Information Retrieval, assignment 3}
\author{Borislav Todorov (4181840) and Zmitser Zhaleznichenka (4134575)}
\date{\today}
\maketitle

\section{Exercise 16.1}

According to the \emph{Cluster hypothesis} "Documents in the same cluster behave similarly with respect to relevance to information needs". In this exercise we will show that this hypothesis does not hold for any notion of similarity.

It is clear that the validity of the cluster hypothesis depends on the effectiveness of the clustering. In this exercise we define two documents as similar(they will end up in the same cluster) if they have at least two names in common. Bellow, you can find the contents of two sample documents which according to the criteria are similar and therefore will be clustered together.

\begin{enumerate}
	\item Barack Obama is the current president of USA. He has invited Nicolas Sarkozy to visit the White House.
	\item Nicolas Sarkozy is the current president of France. He was invited on a visit by Barack Obama.
\end{enumerate}

Both documents have the names "Barack Obama" and "Nicolas Sarkozy" in their content. As a result, they are in the same cluster. Now suppose we want to retrieve the current president of a particular country, for example USA. In this case, it is clear that only the first document is relevant. However, this is against the cluster hypothesis because the two documents in the cluster do not have similar relevance according to the information needs. The conclusion is that the defined clustering criteria is not adequate for the information needs.

\section{Exercise 16.2}

In this exercise we will show that even though clustering can significantly speed up search, it can also sometimes be inexact and produce worse results compared to the direct nearest neighbour search. 

The idea of the cluster-based search is to find the clusters that are closest to the query and only search through the documents from these clusters. Figure 1 shows an example clustering of documents. We have two clusters - green and red. The query is represented by a white circle. The green cluster is closer to the query but a document from the red cluster is the one with the best relevance for the query. If we use the cluster-based approach we will only look into the documents from the green cluster and we will end up with the green document closer to the query as the most relevant result. However, if we use the direct nearest neighbour search approach the result will be the red document closer to the query which indeed is the most relevant document. In this situation the direct nearest neighbour approach will produce a better result. The conclusion is that the cluster-based approach may in certain conditions be inexact and decrease the quality of the search.

\begin{figure}[H]
	\centering
	\includegraphics[]{Clustering.png}
	\caption{}
\end{figure}

\section{Exercise 16.3}

It is generally more difficult to cluster 17 pairs of duplicate points than 17 different points. Two points in a pair are equal and will be placed in one cluster. All the other points will have great difference with the points in a pair and will most probably be assigned to other clusters. This may lead to the creation of 17 clusters each having a pair of duplicated points. To avoid this, which allow to set a restriction on K clustering algorithms should be employed. However, if having these algorithms, it is equally difficult to cluster 17 pairs of points as opposed to 17 distinct points. \newline

\(purity(\Omega,C) = (1/34) * (10 + 8 + 6) \approx 0.71\) \newline

\(NMI(\Omega, C) \approx \frac{0.3919}{|1.0951 * 1.0551| / 2} \approx 0.36\)\footnote{Computed automatically using a Python script. All the other computations were done with Wolfram Alpha.} \newline

\(TP = 97, FP = 80, TN = 288, FN = 96\) \newline

\(RI \approx 0.69 \) \newline

\(P = \frac{97}{97 + 80} \approx 0.6862, R = \frac{97}{97 + 96} \approx 0.5026\) \newline

\(F_{5} = \frac{26 * 0.6862 * 0.5026}{25*0.6862 + 0.5026} \approx 0.5078\) \newline

It turned out that $F_{5}$ value increased, all the other values remained almost the same.

According to the assessment and the measures, $F_5$ measure suits best to assess the quality of clusterings as it gives higher value to the clusterings with equal documents which is logical, because having such documents leads to better similarity between the documents within a cluster.

\end{document}